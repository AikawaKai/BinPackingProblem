%%%%%%%%%%%%%%%%%%%%%%%%%%%%%%%%%%%%%%%%%
% Programming/Coding Assignment
% LaTeX Template
%
% This template has been downloaded from:
% http://www.latextemplates.com
%
% Original author:
% Ted Pavlic (http://www.tedpavlic.com)
%
% Note:
% The \lipsum[#] commands throughout this template generate dummy text
% to fill the template out. These commands should all be removed when 
% writing assignment content.
%
% This template uses a Perl script as an example snippet of code, most other
% languages are also usable. Configure them in the "CODE INCLUSION 
% CONFIGURATION" section.
%
%%%%%%%%%%%%%%%%%%%%%%%%%%%%%%%%%%%%%%%%%

%----------------------------------------------------------------------------------------
%	PACKAGES AND OTHER DOCUMENT CONFIGURATIONS
%----------------------------------------------------------------------------------------

\documentclass{article}

\usepackage[utf8]{inputenc}
\usepackage[italian]{babel} 
\usepackage{fancyhdr} % Required for custom headers
\usepackage{lastpage} % Required to determine the last page for the footer
\usepackage{extramarks} % Required for headers and footers
\usepackage[usenames,dvipsnames]{color} % Required for custom colors
\usepackage{graphicx} % Required to insert images
\usepackage{listings} % Required for insertion of code
\usepackage{courier} % Required for the courier font
\usepackage{lipsum} % Used for inserting dummy 'Lorem ipsum' text into the template
\usepackage{amsmath}
\usepackage{algorithm}
\usepackage[noend]{algpseudocode}
\makeatletter
\def\BState{\State\hskip-\ALG@thistlm}
\makeatother

% Margins
\topmargin=-0.45in
\evensidemargin=0in
\oddsidemargin=0in
\textwidth=6.5in
\textheight=9.0in
\headsep=0.25in

\linespread{1.1} % Line spacing

% Set up the header and footer
\pagestyle{fancy}
\lhead{\hmwktitolobreve} % Top left header
\lfoot{\lastxmark} % Bottom left footer
\cfoot{} % Bottom center footer
\rfoot{Page\ \thepage\ of\ \protect\pageref{LastPage}} % Bottom right footer
\renewcommand\headrulewidth{0.4pt} % Size of the header rule
\renewcommand\footrulewidth{0.4pt} % Size of the footer rule

\setlength\parindent{0pt} % Removes all indentation from paragraphs

%----------------------------------------------------------------------------------------
%	CODE INCLUSION CONFIGURATION
%----------------------------------------------------------------------------------------

\definecolor{MyDarkGreen}{rgb}{0.0,0.4,0.0} % This is the color used for comments
\lstloadlanguages{Perl} % Load Perl syntax for listings, for a list of other languages supported see: ftp://ftp.tex.ac.uk/tex-archive/macros/latex/contrib/listings/listings.pdf
\lstset{language=Perl, % Use Perl in this example
        frame=single, % Single frame around code
        basicstyle=\small\ttfamily, % Use small true type font
        keywordstyle=[1]\color{Blue}\bf, % Perl functions bold and blue
        keywordstyle=[2]\color{Purple}, % Perl function arguments purple
        keywordstyle=[3]\color{Blue}\underbar, % Custom functions underlined and blue
        identifierstyle=, % Nothing special about identifiers                                         
        commentstyle=\usefont{T1}{pcr}{m}{sl}\color{MyDarkGreen}\small, % Comments small dark green courier font
        stringstyle=\color{Purple}, % Strings are purple
        showstringspaces=false, % Don't put marks in string spaces
        tabsize=5, % 5 spaces per tab
        %
        % Put standard Perl functions not included in the default language here
        morekeywords={rand},
        %
        % Put Perl function parameters here
        morekeywords=[2]{on, off, interp},
        %
        % Put user defined functions here
        morekeywords=[3]{test},
       	%
        morecomment=[l][\color{Blue}]{...}, % Line continuation (...) like blue comment
        numbers=left, % Line numbers on left
        firstnumber=1, % Line numbers start with line 1
        numberstyle=\tiny\color{Blue}, % Line numbers are blue and small
        stepnumber=5 % Line numbers go in steps of 5
}

% Creates a new command to include a perl script, the first parameter is the filename of the script (without .pl), the second parameter is the caption
\newcommand{\perlscript}[2]{
\begin{itemize}
\item[]\lstinputlisting[caption=#2,label=#1]{#1.pl}
\end{itemize}
}


%----------------------------------------------------------------------------------------
%	NAME AND CLASS SECTION
%----------------------------------------------------------------------------------------


\newcommand{\hmwkDueDate}{\date{\today}} % Due date
\newcommand{\hmwkClass}{Algoritmi Euristici\\One Dimensional Bin Packing Problem} % Course/class
\newcommand{\hmwktitolobreve}{One Bin Packing Problem} %
\newcommand{\hmwkUniversita}{Università degli Studi Di Milano} % Teacher/lecturer
\newcommand{\hmwkAuthorName}{Marco Odore} % Your name

%----------------------------------------------------------------------------------------
%	TITLE PAGE
%----------------------------------------------------------------------------------------

\title{
\vspace{2in}
\textmd{\textbf{\hmwkClass}}\\
\vspace{0.1in}\large{\textit{\hmwkUniversita}}
\vspace{3in}
}

\author{\textbf{\hmwkAuthorName}}
\date{\today} % Insert date here if you want it to appear below your name

%----------------------------------------------------------------------------------------

\begin{document}
\begin{center}
\maketitle
\end{center}

%----------------------------------------------------------------------------------------
%	TABLE OF CONTENTS
%----------------------------------------------------------------------------------------

%\setcounter{tocdepth}{1} % Uncomment this line if you don't want subsections listed in the ToC

\newpage
\tableofcontents
\newpage

%----------------------------------------------------------------------------------------
%	PROBLEM 1
%----------------------------------------------------------------------------------------

% To have just one problem per page, simply put a \clearpage after each problem

\section{Introduzione}
Lo scopo del lavoro è quello di proporre una possibile implementazione in C di diversi metodi euristici applicati al problema del \textit{One Dimensional Bin Packing}, per la ricerca di soluzioni ottime o che comunque vi si avvicinano.

\subsection{One Dimensional Bin Packing}
Dato un multiset di $n$ oggetti $O=\{o_1, o_2, o_3 \dots o_n\}$, ognuno con dimensione $d_i$, lo scopo è quello di minimizzare il numero di contenitori $b_j$ (bin) $M=\{b_1, b_2, b_3 \dots b_{n} \}$, ognuno con dimensione fissata $B$, che contengono tali oggetti.
\newline
\newline
Il problema è soggetto a diversi vincoli:
\begin{itemize}
\item Ogni oggetto deve essere inserito in un solo contenitore.
\item La somma delle dimensioni $d_i$ degli oggetti $o_i$, nel contenitore $b_j$, non deve superare la dimensione del contenitore.
\[
\sum_{o_i \in b_j} d_i \le B
\]
\item Il numero dei contenitori $b_j$ deve essere il minimo possibile. Si cercherà quindi di minimizzare tale funzione:
\[
min \sum_{j=1}^{n} y_j
\]
In cui $y_i$ è una variabile binaria associata agli $n$ possibili contenitori $b_j$ (il caso peggiore contempla un contenitore per ogni oggetto presente nel multi insieme).

Secondo la teoria della complessità, tale problema ha complessità \textit{NP-hard}. Per tale motivo sono state studiate diverse tecniche euristiche, con lo scopo di ottenere un trade-off tra velocità di esecuzione e ottimalità delle soluzioni generate.

\end{itemize}

\section{Euristiche implementate}
Per la risoluzione del problema sono state implementate due principali euristiche costruttive greedy:

\begin{itemize}
\item FirstFit
\item Minimum Bin Slack (MBS)
\end{itemize}

Che poi sono servite da base per altre due meta euristiche:

\begin{itemize}
\item MBS Sampling
\item Variable Neighbour Search (VNS)
\end{itemize} 
\newpage
\subsection{FirstFit}
Tale algoritmo è molto banale, e si basa sull'idea greedy che, scorrendo iterativamente la lista di oggetti, se nel contenitore $b_j$ corrente c'è abbastanza spazio, allora vi si inserisce l'oggetto corrente $o_i$. Altrimenti, se non c'è spazio tra i contenitori attualmente presenti, se ne genera uno nuovo.
\newline
\begin{algorithm}[h]
\caption{FirstFit}\label{FirstFit}
\begin{algorithmic}[1]
\For {obj in objectList} 
\For {bin in binList}
\If {obj fit in bin}
\State Pack object in bin
\State break
\EndIf
\State{\textbf{end if}}
\EndFor
\State{\textbf{end for}}
\If {obj did not fit in any available bin}
\State Create new bin and pack object in it
\EndIf 
\State{\textbf{end if}}
\EndFor
\State{\textbf{end for}}
\end{algorithmic}
\end{algorithm}
\newline
\newline
\newline
Nel caso peggiore (quando ogni oggetto può essere inserito in un solo contenitore) tale algoritmo ha complessità $O(n^2)$.
\newline
\newline
Una variante di questo algoritmo, il \textit{FirstFit Decreasing}, prende in considerazione l'idea che posizionare oggetti grandi sia più difficile che posizionarne di piccoli, e consiste nell' ordinamento decrescente della lista di oggetti del dataset, prima dell'esecuzione del FirstFit. 
\newline
\newline
\subsection{Minimum Bin Slack}
L'MBS è un'euristica greedy orientata sui contenitori. L'algoritmo consiste nel mantenere, ad ogni passo, una lista di oggetti $Z$ non ancora inseriti, con un ordinamento decrescente, ricercando tra questi l'insieme di oggetti che meglio riempiono il contenitore corrente (idealmente non lasciando spazio libero).
\newline
\newline
La ricerca del sottoinsieme di oggetti da inserire nel contenitore corrente, avviene tramite una procedura ricorsiva, la quale testa tutti i possibili sottoinsiemi della lista $Z$. Se durante la ricerca si trova una soluzione che riempie totalmente il contenitore, questa viene interrotta, poiché non ci può essere una soluzione migliore per il contenitore corrente, ma al massimo equivalente (da notare comunque che il sottoinsieme generato potrebbe non essere ottimo per la soluzione globale).
\newline
\begin{algorithm}[h]
\caption{MBSsearch}\label{MBSsearch}
\begin{algorithmic}[2]
\State \textbf{Procedure MBSsearch(q)}
\For {$int \; r=q$ to $n$} 
\State $obj_{r} = Z[r]$
\If{$size(obj_{r}) \le slack(A)$}
\State $A = A \cup \{obj_{r}\}$
\State $MBSsearch(r+1)$
\State $A = A - \{obj_{r}\}$
\If{$slack(A^*) = 0$}
\State $exit$
\EndIf
\State{\textbf{end if}}
\EndIf 
\State{\textbf{end if}}
\EndFor
\State{\textbf{end for}}
\If{$slack(A) < slack(A^*)$}
\State $A^* = A$
\EndIf
\State{\textbf{end if}}
\end{algorithmic}
\end{algorithm}
\newpage
Dato che in linea teorica questa procedura cerca tutte le possibili combinazioni di elementi da inserire in un contenitore, questa ha complessita $O(2^n)$. Nella pratica è possibile velocizzare l'algoritmo con alcuni accorgimenti. Ad esempio, se lo slack del sottoinsieme corrente è più piccolo del più piccolo elemento dell'insieme di oggetti, cioè che $slack(A) < size(obj_{min})$, allora il ciclo for viene saltato, in quanto per il sottoinsieme $A$ non esiste miglioramento possibile.
\newline
\newline
\subsection{MBS Sampling}

\subsection{Variable Neighbour Search}



%----------------------------------------------------------------------------------------

\end{document}